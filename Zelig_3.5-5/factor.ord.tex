\section{\texttt{factor.ord}: Ordinal Data Factor Analysis}
\label{factor.ord}

Given some unobserved explanatory variables and observed ordinal
dependent variables, this model estimates latent factors using a Gibbs
sampler with data augmentation.  For factor analysis for continuous
data, see \Sref{factor.bayes}.  For factor analysis for mixed data
(including both continuous and ordinal variables), see
\Sref{factor.mix}.

\subsubsection{Syntax}
\begin{verbatim}
> z.out <- zelig(cbind(Y1 ,Y2, Y3) ~ NULL, factors = 1, 
                model = "factor.ord", data = mydata)
\end{verbatim}

\subsubsection{Inputs}
{\tt zelig()} accepts the following arguments for {\tt factor.ord}:  :
\begin{itemize}
\item \texttt{Y1, Y2}, and \texttt{Y3}: variables of interest in
factor analysis (manifest variables), assumed to be ordinal
variables. The number of manifest variables must be greater than the
number of the factors.

\item \texttt{factors}: number of the factors to be fitted (defaults
to 1).
\end{itemize}

\subsubsection{Additional Inputs}

In addition, {\tt zelig()} accepts the following arguments for model
specification:  
\begin{itemize}
\item \texttt{lambda.constraints}: list that contains the equality or 
inequality constraints on the factor loadings. A typical entry in the list
has one of the following forms: 
\begin{itemize}
\item {\tt varname = list()}: by default, no constraints are imposed.
\item \texttt{varname = list(d, c)}: constrains the
$d$th loading for the variable named \texttt{varname} to be equal to \texttt{c};
\item \texttt{varname = list(d, "+")}: constrains the
$d$th loading for the variable named \texttt{varname} to be positive;
\item \texttt{varname = list(d, "-")}: constrains the
$d$th loading for the variable named \texttt{varname} to be negative. 
\end{itemize}
%Unlike \texttt{factanal} the $\Lambda$ matrix 
%has \texttt{factors+1} columns. 
The first column of $\Lambda$ should not be constrained in general.

\item \texttt{drop.constantvars}: defaults to {\tt TRUE}, dropping the
manifest variables that have no variation before fitting the model. 

\end{itemize}

The model accepts the following arguments to monitor the convergence
of the Markov chain:  
\begin{itemize}
\item \texttt{burnin}: number of the initial MCMC iterations to be 
 discarded (defaults to 1,000).

\item \texttt{mcmc}: number of the MCMC iterations after burnin
(defaults to 20,000).

\item \texttt{thin}: thinning interval for the Markov chain. Only every 
 \texttt{thin}-th draw from the Markov chain is kept. The value of
\texttt{mcmc} must be divisible by this value. The default value is 1.

\item \texttt{tune}: tuning parameter for Metropolis-Hasting sampling,
either a scalar or a vector of length $K$. The value of the tuning
parameter must be positive. The default value is 1.2.

\item \texttt{verbose}: defaults to {\tt FALSE}. If \texttt{TRUE}, the
progress of the sampler (every $10\%$) is printed to the screen.

\item \texttt{seed}: seed for the random number generator. The 
default is \texttt{NA} which corresponds to a random seed 12345. 

\item \texttt{Lambda.start}: starting values of the factor loading
matrix $\Lambda$ for the Markov chain, either a scalar (all
unconstrained loadings are set to that value), or a matrix with
compatible dimensions.  The default is {\tt NA}, such that the start
values for the first column are set based on the observed pattern,
while the remaining columns have start values set to 0 for
unconstrained factor loadings, and -1 or 1 for constrained loadings
(depending on the nature of the constraints).  

\item \texttt{store.lambda}: defaults to {\tt TRUE}, which stores the
posterior draws of the factor loadings.  

\item \texttt{store.scores}: defaults to {\tt FALSE}.  If {\tt TRUE},
stores the posterior draws of the factor scores.  (Storing factor
scores may take large amount of memory for a a large number of draws
or observations.)

\end{itemize}

Use the following parameters to specify the model's priors:  
\begin{itemize}
\item \texttt{l0}: mean of the Normal prior for the factor
loadings, either a scalar or a matrix with the same dimensions as
$\Lambda$.  If a scalar value, that value will be the prior mean for
all the factor loadings. Defaults to 0.

\item \texttt{L0}: precision parameter of the Normal prior 
for the factor loadings, either a scalar or a matrix with the same
dimensions as $\Lambda$.  If \texttt{L0} takes a scalar value, then
the precision matrix will be a diagonal matrix with the diagonal
elements set to that value. The default value is 0, which leads to an
improper prior.
\end{itemize}

Zelig users may wish to refer to \texttt{help(MCMCordfactanal)} for more 
information.

\subsubsection{Convergence}

Users should verify that the Markov Chain converges to its stationary 
distribution.  After running the \texttt{zelig()} function but before 
performing \texttt{setx()}, users may conduct the following 
convergence diagnostics tests:

\begin{itemize}
\item \texttt{geweke.diag(z.out\$coefficients)}: The Geweke diagnostic tests
the null hypothesis that the Markov chain is in the stationary distribution
and produces z-statistics for each estimated parameter. 
 
\item \texttt{heidel.diag(z.out\$coefficients)}: The Heidelberger-Welch
diagnostic first tests the null hypothesis that the Markov Chain is in the
stationary distribution and produces p-values for each estimated parameter. 
 Calling \texttt{heidel.diag()} also produces
output that indicates whether the mean of a marginal posterior distribution
can be estimated with sufficient precision, assuming that the Markov Chain is
in the stationary distribution.

\item \texttt{raftery.diag(z.out\$coefficients)}: The Raftery diagnostic
indicates how long the Markov Chain should run before considering draws from
the marginal posterior distributions sufficiently representative of the
stationary distribution.
\end{itemize}

\noindent If there is evidence of non-convergence, adjust the values 
for \texttt{burnin} and \texttt{mcmc} and rerun \texttt{zelig()}.

Advanced users may wish to refer to \texttt{help(geweke.diag)}, 
\texttt{help(heidel.diag)}, and \texttt{help(raftery.diag)} for more 
information about these diagnostics. 


\subsubsection{Examples}

\begin{enumerate}
\item {Basic Example} \\
Attaching the sample  dataset:
\begin{Schunk}
\begin{Sinput}
RRR>  data(newpainters)
\end{Sinput}
\end{Schunk}

Factor analysis for ordinal data using \texttt{factor.ord}:
\begin{Schunk}
\begin{Sinput}
RRR>  z.out <- zelig(cbind(Composition,Drawing,Colour,Expression)~NULL,
+                     data=newpainters, model="factor.ord",
+                     factors=1, L0=0.5,
+                     burin=5000,mcmc=30000, thin=5, tune=1.2,verbose=TRUE)
RRR>                     
\end{Sinput}
\end{Schunk}
Checking for convergence before summarizing the estimates:
\begin{Schunk}
\begin{Sinput}
RRR>  geweke.diag(z.out$coefficients)
\end{Sinput}
\end{Schunk}
\begin{Schunk}
\begin{Sinput}
RRR>  heidel.diag(z.out$coefficients)
\end{Sinput}
\end{Schunk}
\begin{Schunk}
\begin{Sinput}
RRR>  raftery.diag(z.out$coefficients)
\end{Sinput}
\end{Schunk}
\begin{Schunk}
\begin{Sinput}
RRR>  summary(z.out) 
\end{Sinput}
\end{Schunk}


\end{enumerate}

\subsubsection{Model}

Let $Y_i$ be a vector of $K$ observed ordinal variables for
observation $i$, each ordinal variable $k$ for $k=1,\ldots, K$ takes
integer value $j=1, \ldots, J_k$. The distribution of $Y_i$ is assumed
to be governed by another $k$-vector of unobserved continuous variable
$Y_i^*$. There are $d$ underlying factors.

\begin{itemize}
\item The \emph{stochastic component} is described in terms of the latent
variable $Y_i^*$:
\begin{eqnarray*}
Y_{i}^*  \sim \textrm{Normal}_K(\mu_i, I_K),
\end{eqnarray*}
where $Y_i^*=(Y_{i1}^*, \ldots, Y_{iK}^*)$, and $\mu_i$ is the mean vector
for $Y_i^*$, and $\mu_i=(\mu_{i1},\ldots, \mu_{iK})$.

Instead of $Y_{ik}^*$, we observe ordinal  variable $Y_{ik}$,
\begin{eqnarray*}
Y_{ik} = j \textrm{ if } \gamma_{(j-1),k} \le Y_{ik}^* \le \gamma_{jk}
\textrm{ for } \quad j=1,\ldots, J_k, k=1,\ldots, K.
\end{eqnarray*}
where $\gamma_{jk}, j=0,\ldots, J$ are the threshold parameters for
the $k$th variable with the following constraints, $\gamma_{lk} <
\gamma_{mk}$ for $l < m$, and $\gamma_{0k}=-\infty, \gamma_{J_k
k}=\infty$ for any $k=1, \ldots, K$.  It follows that the probability
of observing $Y_{ik}$ belonging to category $j$ is,
\begin{eqnarray*}
\Pr(Y_{ik}=j) =\Phi(\gamma_{jk} \mid \mu_{ik})-\Phi(\gamma_{(j-1),k} \mid \mu_{ik}) \textrm{ for } j=1,\ldots,J_k
\end{eqnarray*}
where $\Phi(\cdot \mid\mu_{ik})$ is the cumulative distribution
function of the Normal distribution with mean $\mu_{ik}$ and variance
1.

\item The \emph{systematic component} is given by,
\begin{eqnarray*}
\mu_i = \Lambda\phi_i,
\end{eqnarray*}
where $\Lambda$ is a $K \times d$ matrix of factor loadings for each
variable, $\phi_i$ is a $d$-vector of factor scores for observation
$i$. Note both $\Lambda$ and $\phi$ need to be estimated.

\item The independent conjugate \emph{prior} for each element of $\Lambda$,
$\Lambda_{ij}$ is given by
\begin{eqnarray*}
\Lambda_{ij} \sim \textrm{Normal}(l_{0_{ij}}, L_{0_{ij}}^{-1})
\textrm{ for } i=1,\ldots, k; \quad j=1,\ldots, d.
\end{eqnarray*}

\item The \emph{prior} for $\phi_i$ is,
\begin{eqnarray*}
\phi_{i(2:d)} \sim \textrm{Normal}(0, I_{d-1}), \textrm{ for } \quad i=2, \ldots, n.
\end{eqnarray*}
where $I_{d-1}$ is a $ (d-1)\times (d-1) $ identity matrix. Note the
first element of $\phi_i$ is 1.

\end{itemize}

\subsubsection{Output Values}

The output of each Zelig command contains useful information which you
may view. For example, if you run:
\begin{verbatim}
z.out <- zelig(cbind(Y1, Y2, Y3), model = "factor.ord", data)
\end{verbatim}

\noindent then you may examine the available information in \texttt{z.out} by
using \texttt{names(z.out)}, see the draws from the posterior
distribution of the \texttt{coefficients} by using
\texttt{z.out\$coefficients}, and view a default summary of
information through \texttt{summary(z.out)}. Other elements available
through the \texttt{\$} operator are listed below.

\begin{itemize}
\item From the \texttt{zelig()} output object \texttt{z.out}, you may extract:

\begin{itemize}
\item \texttt{coefficients}: draws from the posterior distributions
of the estimated factor loadings, the estimated cut points $\gamma$ for each
variable. Note the first element of $\gamma$ is normalized to be 0. If 
\texttt{store.scores=TRUE}, the estimated factors scores are also contained in 
\texttt{coefficients}.

   \item {\tt zelig.data}: the input data frame if {\tt save.data = TRUE}.  

\item \texttt{seed}: the random seed used in the model.   

\end{itemize}

\item Since there are no explanatory variables, the \texttt{sim()} procedure is
not applicable for factor analysis models.

\end{itemize}

\subsection* {How to Cite} 

To cite the \emph{ factor.ord } Zelig model:
 \begin{verse}
 Ben Goodrich and Ying Lu. 2007. "factor.ord: Ordinal Data Factor Analysis" in Kosuke Imai, Gary King, and Olivia Lau, "Zelig: Everyone's Statistical Software,"\url{http://gking.harvard.edu/zelig} 
\end{verse}
\input{citeZelig}
