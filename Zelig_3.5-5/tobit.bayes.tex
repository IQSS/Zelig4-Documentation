\section{\texttt{tobit.bayes}: Bayesian Linear Regression for a
Censored Dependent Variable} \label{tobit.bayes}

Bayesian tobit regression estimates a linear regression model with a
censored dependent variable using a Gibbs sampler.  The dependent
variable may be censored from below and/or from above.  For other
linear regression models with fully observed dependent variables, see
Bayesian regression (\Sref{normal.bayes}), maximum likelihood normal
regression (\Sref{normal}), or least squares (\Sref{ls}).

\subsubsection{Syntax}
\begin{verbatim}
> z.out <- zelig(Y ~ X1 + X2, below = 0, above = Inf, 
                  model = "tobit.bayes", data = mydata)
> x.out <- setx(z.out)
> s.out <- sim(z.out, x = x.out)
\end{verbatim}

\subsubsection{Inputs}
{\tt zelig()} accepts the following arguments to specify how the
dependent variable is censored.
\begin{itemize}
\item \texttt{below}: point at which the dependent variable is censored
from below. If the dependent variable is only censored from above, set 
\texttt{below = -Inf}. The default value is 0.
\item \texttt{above}: point at which the dependent variable is censored
from above. If the dependent variable is only censored from below, set 
\texttt{above = Inf}. The default value is \texttt{Inf}.
\end{itemize}

\subsubsection{Additional Inputs}

Use the following arguments to monitor the convergence of the Markov
chain:  
\begin{itemize}
\item \texttt{burnin}: number of the initial MCMC iterations to be 
 discarded (defaults to 1,000).

\item \texttt{mcmc}: number of the MCMC iterations after burnin
(defaults to 10,000).

\item \texttt{thin}: thinning interval for the Markov chain. Only every 
 \texttt{thin}-th draw from the Markov chain is kept. The value of
\texttt{mcmc} must be divisible by this value. The default value is 1.

\item \texttt{verbose}: defaults to {\tt FALSE}.  If \texttt{TRUE},
the progress of the sampler (every $10\%$) is printed to the screen.

\item \texttt{seed}: seed for the random number generator. The default is 
\texttt{NA} which corresponds to a random seed of 12345. 

\item \texttt{beta.start}: starting values for the Markov 
chain, either a scalar or vector with length equal to the number 
of estimated coefficients. The default is \texttt{NA}, such that the
least squares estimates are used as the starting values.  

\end{itemize}

Use the following parameters to specify the model's priors:  
\begin{itemize}
\item \texttt{b0}: prior mean for the coefficients, either a numeric
vector or a scalar. If a scalar, that value will be the prior mean for
all coefficients. The default is 0.

\item \texttt{B0}: prior precision parameter for the coefficients,
either a square matrix (with the dimensions equal to the number of the
coefficients) or a scalar. If a scalar, that value times an identity
matrix will be the prior precision parameter. The default is 0, which
leads to an improper prior.

\item \texttt{c0}: \texttt{c0/2} is the shape parameter for the Inverse Gamma
prior on the variance of the disturbance terms. 

\item \texttt{d0}: \texttt{d0/2} is the scale parameter for the Inverse Gamma
prior on the variance of the disturbance terms. 

\end{itemize}

Zelig users may wish to refer to \texttt{help(MCMCtobit)} for more 
information.

\subsubsection{Convergence}

Users should verify that the Markov Chain converges to its stationary 
distribution.  After running the \texttt{zelig()} function but before 
performing \texttt{setx()}, users may conduct the following 
convergence diagnostics tests:

\begin{itemize}
\item \texttt{geweke.diag(z.out\$coefficients)}: The Geweke diagnostic tests
the null hypothesis that the Markov chain is in the stationary distribution
and produces z-statistics for each estimated parameter. 
 
\item \texttt{heidel.diag(z.out\$coefficients)}: The Heidelberger-Welch
diagnostic first tests the null hypothesis that the Markov Chain is in the
stationary distribution and produces p-values for each estimated parameter. 
 Calling \texttt{heidel.diag()} also produces
output that indicates whether the mean of a marginal posterior distribution
can be estimated with sufficient precision, assuming that the Markov Chain is
in the stationary distribution.

\item \texttt{raftery.diag(z.out\$coefficients)}: The Raftery diagnostic
indicates how long the Markov Chain should run before considering draws from
the marginal posterior distributions sufficiently representative of the
stationary distribution.
\end{itemize}

\noindent If there is evidence of non-convergence, adjust the values 
for \texttt{burnin} and \texttt{mcmc} and rerun \texttt{zelig()}.

Advanced users may wish to refer to \texttt{help(geweke.diag)}, 
\texttt{help(heidel.diag)}, and \texttt{help(raftery.diag)} for more 
information about these diagnostics. 


\subsubsection{Examples}

\begin{enumerate}
\item {Basic Example} \\
Attaching the sample  dataset:
\begin{Schunk}
\begin{Sinput}
RRR>  data(tobin)
\end{Sinput}
\end{Schunk}
Estimating linear regression using \texttt{tobit.bayes}:
\begin{Schunk}
\begin{Sinput}
RRR>  z.out <- zelig(durable ~ age + quant, model = "tobit.bayes",
+                   data = tobin, verbose=TRUE)
\end{Sinput}
\end{Schunk}
Checking for convergence before summarizing the estimates:
\begin{Schunk}
\begin{Sinput}
RRR>  geweke.diag(z.out$coefficients)
\end{Sinput}
\end{Schunk}
\begin{Schunk}
\begin{Sinput}
RRR> heidel.diag(z.out$coefficients)
\end{Sinput}
\end{Schunk}
\begin{Schunk}
\begin{Sinput}
RRR> raftery.diag(z.out$coefficients)
\end{Sinput}
\end{Schunk}
\begin{Schunk}
\begin{Sinput}
RRR> summary(z.out)
\end{Sinput}
\end{Schunk}
Setting values for the explanatory variables to their sample averages:
\begin{Schunk}
\begin{Sinput}
RRR>  x.out <- setx(z.out)
\end{Sinput}
\end{Schunk}
Simulating quantities of interest from the posterior distribution given 
\texttt{x.out}.
\begin{Schunk}
\begin{Sinput}
RRR>  s.out1 <- sim(z.out, x = x.out)
\end{Sinput}
\end{Schunk}
\begin{Schunk}
\begin{Sinput}
RRR> summary(s.out1)
\end{Sinput}
\end{Schunk}
\item {Simulating First Differences} \\
Set explanatory variables to their default(mean/mode) values, with high
(80th percentile) and low (20th percentile) liquidity ratio (\texttt{quant}):

\begin{Schunk}
\begin{Sinput}
RRR>  x.high <- setx(z.out, quant = quantile(tobin$quant, prob = 0.8))
RRR>  x.low <- setx(z.out, quant = quantile(tobin$quant, prob = 0.2))
\end{Sinput}
\end{Schunk}
Estimating the first difference for the effect of
high versus low liquidity ratio on duration(\texttt{durable}):
\begin{Schunk}
\begin{Sinput}
RRR>  s.out2 <- sim(z.out, x = x.high, x1 = x.low)
\end{Sinput}
\end{Schunk}
\begin{Schunk}
\begin{Sinput}
RRR> summary(s.out2)
\end{Sinput}
\end{Schunk}
\end{enumerate}

\subsubsection{Model}
Let $Y_i^*$ be the dependent variable which is not directly observed. Instead,
we observe $Y_i$ which is defined as following:
\begin{equation*}
Y_i = \left\{
\begin{array}{lcl}
Y_i^*  &\textrm{if} & c_1<Y_i^*<c_2 \\
c_1    &\textrm{if} & c_1 \ge Y_i^* \\
c_2    &\textrm{if} & c_2 \le Y_i^*
\end{array}\right.
\end{equation*}
where $c_1$ is the lower bound below which $Y_i^*$ is censored, and
$c_2$ is the upper bound above which $Y_i^*$ is censored.

\begin{itemize}
\item The \emph{stochastic component} is given by 
\begin{eqnarray*} 
\epsilon_{i}  &  \sim & \textrm{Normal}(0, \sigma^2)
\end{eqnarray*} 
where $\epsilon_{i}=Y^*_i-\mu_i$. 

\item The \emph{systematic component} is given by
\begin{eqnarray*}
\mu_{i}= x_{i} \beta,
\end{eqnarray*}
where $x_{i}$ is the vector of $k$ explanatory variables for
observation $i$ and $\beta$ is the vector of coefficients.

\item The \emph{semi-conjugate priors} for $\beta$ and $\sigma^2$ are
given by 
\begin{eqnarray*} 
\beta & \sim & \textrm{Normal}_k \left( b_{0},B_{0}^{-1}\right) \\
\sigma^{2} & \sim & \textrm{InverseGamma} \left( \frac{c_0}{2}, \frac{d_0}{2}
\right) 
\end{eqnarray*}
where $b_{0}$ is the vector of means for the $k$ explanatory
variables, $B_{0}$ is the $k\times k$ precision matrix (the inverse of
a variance-covariance matrix), and $c_0/2$ and $d_0/2$ are the shape
and scale parameters for $\sigma^{2}$.  Note that $\beta$ and
$\sigma^2$ are assumed \emph{a priori} independent.
\end{itemize}

\subsubsection{Quantities of Interest}

\begin{itemize}
\item The expected values (\texttt{qi\$ev}) for the tobit regression model is
calculated as following.  Let
\begin{eqnarray*}
\Phi_1 &=& \Phi\left(\frac{(c_1 - x \beta)}{\sigma}\right) \\
\Phi_2 &=& \Phi\left(\frac{(c_2 - x \beta)}{\sigma}\right) \\ 
\phi_1 &=& \phi\left(\frac{(c_1 - x \beta)}{\sigma}\right) \\
\phi_2 &=& \phi\left(\frac{(c_2 - x \beta)}{\sigma}\right) 
\end{eqnarray*}
where $\Phi(\cdot)$ is the (cumulative) Normal density function and
$\phi(\cdot)$ is the Normal probability density function of the
standard normal distribution.  Then the expected values are
\begin{eqnarray*}
E(Y|x) &=& P(Y^* \le c_1|x) c_1+P(c_1<Y^*<c_2|x) E(Y^* \mid c_1<Y^*<c_2, x)+P(Y^* \ge c_2) c_2 \\
   &=& \Phi_{1}c_1 + x \beta(\Phi_{2}-\Phi_{1}) + \sigma (\phi_1 -\phi_2) + (1-\Phi_2) c_2,
\end{eqnarray*}

\item The first difference (\texttt{qi\$fd}) for the tobit regression
model is defined as
\begin{eqnarray*}
\text{FD}=E(Y\mid x_{1})-E(Y\mid x).
\end{eqnarray*}

\item In conditional prediction models, the average expected treatment effect
(\texttt{qi\$att.ev}) for the treatment group is
\begin{eqnarray*}
\frac{1}{\sum t_{i}}\sum_{i:t_{i}=1}[Y_{i}(t_{i}=1)-E[Y_{i}(t_{i}=0)]],
\end{eqnarray*}
where $t_{i}$ is a binary explanatory variable defining the treatment
($t_{i}=1$) and control ($t_{i}=0$) groups. 

\end{itemize}

\subsubsection{Output Values}

The output of each Zelig command contains useful information which you may
view. For example, if you run:
\begin{verbatim}
z.out <- zelig(y ~ x, model = "tobit.bayes", data)
\end{verbatim}

\noindent then you may examine the available information in \texttt{z.out} by
using \texttt{names(z.out)}, see the draws from the posterior distribution of
the \texttt{coefficients} by using \texttt{z.out\$coefficients}, and view a 
default summary of information through \texttt{summary(z.out)}. Other elements
available through the \texttt{\$} operator are listed below.

\begin{itemize}
\item From the \texttt{zelig()} output object \texttt{z.out}, you may extract:

\begin{itemize}
\item \texttt{coefficients}: draws from the posterior distributions
of the estimated parameters. The first $k$ columns contain the posterior draws
of the coefficients $\beta$, and the last column contains the posterior draws 
of the variance $\sigma^2$.

   \item {\tt zelig.data}: the input data frame if {\tt save.data = TRUE}.  
\item \texttt{seed}: the random seed used in the model.

\end{itemize}

\item From the \texttt{sim()} output object \texttt{s.out}:

\begin{itemize}
\item \texttt{qi\$ev}: the simulated expected value for the specified
values of \texttt{x}.

\item \texttt{qi\$fd}: the simulated first difference in the expected
values given the values specified in \texttt{x} and \texttt{x1}.

\item \texttt{qi\$att.ev}: the simulated average expected treatment effect
for the treated from conditional prediction models.

\end{itemize}
\end{itemize}

\subsection* {How to Cite} 

To cite the \emph{ tobit.bayes } Zelig model:
 \begin{verse}
 Ben Goodrich and Ying Lu. 2007. "tobit.bayes: Bayesian Linear Regression for a Censored Dependent Variable" in Kosuke Imai, Gary King, and Olivia Lau, "Zelig: Everyone's Statistical Software,"\url{http://gking.harvard.edu/zelig} 
\end{verse}
\input{citeZelig}
\subsection*{See also}
Bayesian tobit regression is part of the MCMCpack library by Andrew D. Martin and Kevin M. Quinn \citep{MarQui05}. The convergence diagnostics are part of the CODA library by Martyn Plummer, Nicky Best, Kate Cowles, and Karen Vines \citep{PluBesCowVin05}.
