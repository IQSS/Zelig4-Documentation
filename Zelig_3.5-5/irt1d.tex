\section{\texttt{irt1d}: One Dimensional Item Response Model}

\label{irt1d}

Given several observed dependent variables and an unobserved
explanatory variable, item response theory estimates the latent
variable (ideal points).  The model is estimated using the Markov
Chain Monte Carlo algorithm via a Gibbs sampler and data augmentation.
Use this model if you believe that the ideal points lie in one
dimension, and see the $k$-dimensional item response model
(\Sref{irtkd}) for $k$ hypothesized latent variables.

\subsubsection{Syntax}
\begin{verbatim}
> z.out <- zelig(cbind(Y1, Y2, Y3) ~ NULL, model = "irt1d", data = mydata)
\end{verbatim}

\subsubsection{Inputs}
\texttt{irt1d} accepts the following argument:
\begin{itemize}
\item \texttt{Y1, Y2}, and \texttt{Y3}: \texttt{Y1} contains the items for 
subject ``Y1'', \texttt{Y2} contains the items for subject ``Y2'', and
so on.
\end{itemize}

\subsubsection{Additional arguments}

\texttt{irt1d} accepts the following additional arguments for model specification:

\begin{itemize}
\item \texttt{theta.constraints}: a list specifying possible equality
or inequality constraints on the ability parameters $\theta$.  A
typical entry takes one of the following forms:  

\begin{itemize}

\item {\tt varname = list()}: by default, no constraints are
imposed.

\item \texttt{varname = c}:  constrains the ability parameter for the subject named 
\texttt{varname} to be equal to \texttt{c}. 

\item \texttt{varname = "+"}: constrains the ability parameter for the subject named 
\texttt{varname} to be positive. 

\item \texttt{varname = "-"}: constrains the ability parameter for the
subject named {\tt varname} to be negative.

\end{itemize}
\end{itemize} 

The model also accepts the following arguments to monitor the sampling
scheme for the Markov chain:

\begin{itemize}
\item \texttt{burnin}: number of the initial MCMC iterations to be 
 discarded (defaults to 1,000).

\item \texttt{mcmc}: number of the MCMC iterations after burnin
(defaults to 20,000).

\item \texttt{thin}: thinning interval for the Markov chain. Only every 
 \texttt{thin}-th draw from the Markov chain is kept. The value of 
\texttt{mcmc} must be divisible by this value. The default value is 1.

\item \texttt{verbose}: defaults to {\tt FALSE}. If \texttt{TRUE}, the progress 
 of the sampler (every $10\%$) is printed to the screen. 

\item \texttt{seed}: seed for the random number generator. The default
is \texttt{NA} which corresponds to a random seed 12345.

\item \texttt{theta.start}: starting values for the subject abilities
(ideal points), either a scalar or a vector with length equal to the
number of subjects. If a scalar, that value will be the starting value
for all subjects. The default is \texttt{NA}, which sets the starting
values based on an eigenvalue-eigenvector decomposition of the
agreement score matrix formed from the model response matrix
(\texttt{cbind(Y1, Y2, ...)}).

\item \texttt{alpha.start}: starting values for the difficulty
parameters $\alpha$, either a scalar or a vector with length equal to
the number of the items. If a scalar, the value will be the starting
value for all $\alpha$. The default is \texttt{NA}, which sets the
starting values based on a series of probit regressions that condition
on \texttt{theta.start}.

\item \texttt{beta.start}: starting values for the $\beta$ discrimination
parameters, either a scalar or a vector with length equal to the
number of the items. If a scalar, the value will be the starting value
for all $\beta$. The default is \texttt{NA}, which sets the starting
values based on a series of probit regressions conditioning on
\texttt{theta.start}.

\item \texttt{store.item}: defaults to {\tt TRUE}, storing the
posterior draws of the item parameters.  (For a large number of draws or
a large number observations, this may take a lot of memory.)  

\item \texttt{drop.constant.items}: defaults to {\tt TRUE}, dropping
items with no variation before fitting the model.  

\end{itemize}

\noindent \texttt{irt1d} accepts the following additional arguments to 
specify prior parameters used in the model:

\begin{itemize}

\item \texttt{t0}: prior mean of the subject abilities
(ideal points). The default is 0.

\item \texttt{T0}: prior precision of the subject abilities
(ideal points). The default is 0.

\item \texttt{ab0}: prior mean of $(\alpha, \beta)$. It can be a scalar or
a vector of length 2. If it takes a scalar value, then the prior means for
both $\alpha$ and $\beta$ will be set to that value. The default is 0. 

\item \texttt{AB0}: prior precision of $(\alpha, \beta)$. It can be 
a scalar or a $2 \times 2$ matrix. If it takes a scalar value, 
then the prior precision will be \texttt{diag(AB0,2)}. The prior precision
is assumed to be same for all the items. The default is 0.25.

\end{itemize}

Zelig users may wish to refer to \texttt{help(MCMCirt1d)} for more 
information.

\subsubsection{Convergence}

Users should verify that the Markov Chain converges to its stationary 
distribution.  After running the \texttt{zelig()} function but before 
performing \texttt{setx()}, users may conduct the following 
convergence diagnostics tests:

\begin{itemize}
\item \texttt{geweke.diag(z.out\$coefficients)}: The Geweke diagnostic tests
the null hypothesis that the Markov chain is in the stationary distribution
and produces z-statistics for each estimated parameter. 
 
\item \texttt{heidel.diag(z.out\$coefficients)}: The Heidelberger-Welch
diagnostic first tests the null hypothesis that the Markov Chain is in the
stationary distribution and produces p-values for each estimated parameter. 
 Calling \texttt{heidel.diag()} also produces
output that indicates whether the mean of a marginal posterior distribution
can be estimated with sufficient precision, assuming that the Markov Chain is
in the stationary distribution.

\item \texttt{raftery.diag(z.out\$coefficients)}: The Raftery diagnostic
indicates how long the Markov Chain should run before considering draws from
the marginal posterior distributions sufficiently representative of the
stationary distribution.
\end{itemize}

\noindent If there is evidence of non-convergence, adjust the values 
for \texttt{burnin} and \texttt{mcmc} and rerun \texttt{zelig()}.

Advanced users may wish to refer to \texttt{help(geweke.diag)}, 
\texttt{help(heidel.diag)}, and \texttt{help(raftery.diag)} for more 
information about these diagnostics. 


\subsubsection{Examples}

\begin{enumerate}
\item {Basic Example} \\
Attaching the sample  dataset:
\begin{Schunk}
\begin{Sinput}
RRR>  data(SupremeCourt)
RRR> names(SupremeCourt) <- c("Rehnquist","Stevens","OConnor","Scalia",
+                          "Kennedy","Souter","Thomas","Ginsburg","Breyer")
RRR> 
\end{Sinput}
\end{Schunk}
Fitting a one-dimensional item response theory model using \texttt{irt1d}:
\begin{Schunk}
\begin{Sinput}
RRR>  z.out <- zelig(cbind(Rehnquist, Stevens, OConnor, Scalia, Kennedy,
+                        Souter, Thomas, Ginsburg, Breyer) ~ NULL,
+                data = SupremeCourt, model = "irt1d",
+                B0.alpha = 0.2, B0.beta = 0.2, burnin = 500, mcmc = 10000,
+                thin = 20, verbose = TRUE)
\end{Sinput}
\end{Schunk}

Checking for convergence before summarizing the estimates:
\begin{Schunk}
\begin{Sinput}
RRR>  geweke.diag(z.out$coefficients)
\end{Sinput}
\end{Schunk}
\begin{Schunk}
\begin{Sinput}
RRR> heidel.diag(z.out$coefficients)
\end{Sinput}
\end{Schunk}
\begin{Schunk}
\begin{Sinput}
RRR>  summary(z.out)
\end{Sinput}
\end{Schunk}
\end{enumerate}

\subsubsection{Model}

Let $Y_i$ be a vector of choices on $J$ items made by subject $i$ for
$i=1, \ldots, n$. The choice $Y_{ij}$ is assumed to be determined by
an unobserved utility $Z_{ij}$, which is a function of the subject $i$'s
abilities (ideal points) $\theta_i$ and item parameters $\alpha_j$ and
$\beta_j$ as follows:  
\begin{eqnarray*}
Z_{ij} &=& -\alpha_j + \beta_j' \theta_i + \epsilon_{ij}.
\end{eqnarray*}

\begin{itemize}
\item The \emph{stochastic component} is given by
\begin{eqnarray*}
Y_{ij}  &  \sim & \textrm{Bernoulli}(\pi_{ij}) \\
& = & \pi_{ij}^{Y_{ij}}(1-\pi_{ij})^{1-Y_{ij}},
\end{eqnarray*}
where $\pi_{ij}=\Pr(Y_{ij}=1)=E(Z_{ij})$.

The error term in the unobserved utility equation is independently
and identically distributed with 
\begin{eqnarray*}
\epsilon_{ij} \sim \textrm{Normal}(0, 1).
\end{eqnarray*}

\item The \emph{systematic component} is given by

\begin{eqnarray*}
\pi_{ij}= \Phi(-\alpha_j + \beta_j' \theta_i),
\end{eqnarray*}
where $\Phi(\cdot)$ is the cumulative density function of the standard
normal distribution with mean 0 and variance 1, $\theta_i$ is the
subject ability (ideal point) parameter, and $\alpha_j$ and $\beta_j$
are the item parameters. Both subject abilities and item parameters
are estimated from the model, such that the model is identified by
placing constraints on the subject ability parameters.

\item The \emph{prior} for $\theta_i$ is given by
\begin{eqnarray*}
\theta_i &\sim& \textrm{Normal} \left(  t_{0},T_{0}^{-1}\right)
\end{eqnarray*}

\item The joint \emph{prior} for $\alpha_j$ and $\beta_j$ is given by
\begin{eqnarray*}
(\alpha_j, \beta_j)' &\sim& \textrm{Normal}\left(  ab_{0},AB_{0}^{-1}\right)
\end{eqnarray*}
where $ab_0$ is a 2-vector of prior means and $AB_0$ is a $2 \times 2$ prior
precision matrix.

\end{itemize}
 
\subsubsection{Output Values}

The output of each Zelig command contains useful information which you may
view. For example, if you run:
\begin{verbatim}
z.out <- zelig(cbind(Y1, Y2, Y3) ~ NULL, model = "irt1d", data)
\end{verbatim}
\noindent then you may examine the available information in \texttt{z.out} by
using \texttt{names(z.out)}, see the draws from the posterior distribution of
the \texttt{coefficients} by using \texttt{z.out\$coefficients}, and view 
a default summary of information through \texttt{summary(z.out)}. 
Other elements available through the \texttt{\$} operator are listed below.

\begin{itemize}

\item From the \texttt{zelig()} output object \texttt{z.out}, you may extract:

\begin{itemize}
\item \texttt{coefficients}: draws from the posterior distributions
of the estimated subject abilities(ideal points). If
\texttt{store.item = TRUE}, the estimated item parameters $\alpha$ and
$\beta$ are also contained in \texttt{coefficients}.

\item \texttt{data}: the name of the input data frame.
\item \texttt{seed}: the random seed used in the model.   

\end{itemize}

\item Since there are no explanatory variables, the \texttt{sim()}
procedure is not applicable for item response models.

\end{itemize}

\subsection*{How to Cite}

To cite the \emph{ irt1d } Zelig model:
 \begin{verse}
 Ben Goodrich and Ying Lu. 2007. "irt1d: One Dimensional Item Response Model" in Kosuke Imai, Gary King, and Olivia Lau, "Zelig: Everyone's Statistical Software,"\url{http://gking.harvard.edu/zelig} 
\end{verse}
\input{citeZelig}
\subsection*{See also}
The unidimensional item-response function is part of the MCMCpack
library by Andrew D. Martin and Kevin M. Quinn \citep{MarQui05}.  The
convergence diagnostics are part of the CODA library by Martyn
Plummer, Nicky Best, Kate Cowles, and Karen Vines
\citep{PluBesCowVin05}. Sample data are adapted from \cite{MarQui05}.
