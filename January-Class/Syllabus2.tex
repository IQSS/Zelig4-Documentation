\documentclass{article}


\title{Programming Statistical Models with Zelig 4}
\author{Matt Owen}


\begin{document}


\maketitle


% Introduction
%
\section{Introduction}
\label{intro}

In this course, students will develop statistical packages in the R programming 
language, using the Zelig 4 software suite and API. Emphasis is placed on using 
generalized modeling techniques - linear regressions, simulations - to create 
predictive models that work with a large array of data-sets. Students will also 
learn the basic of developing R packages for distribution via CRAN 
(The Comprehensive R Network), a platform for sharing open-source statistical 
software on the internet.


% Prequisites
%
\section{Prerequisites}
\label{prereq}

Due to the brevity of the course, several requirements are imposed on registered
students:

\begin{itemize}
  \item Basic Programming: conditional statements, loops and writing functions
  \item Basic R Programming: Sampling distributions and Linear Regressions
  \item Statistics: Descriptive statistics, linear regressions and bootstrapping
  \item Math: Matrix algebra
\end{itemize}


% Goals
%
\section{Goals}
\label{goals}

Upon completion of this course, students should have an understanding of the processes
involved in:

\begin{itemize}
	\item Developing statistical packages in R
	\item Programming statistical in Mathematical ideas using Zelig
	\item Submitting packages to CRAN
\end{itemize}


% Schedule
%
\section{Schedule}
\label{schedule}

This course will be divided between 4 days, each with a lecture and workshop component.
Workshops will focus on the material covered during the lecture series with emphasis 
being given to practical application and problem-solving.


% January 17th
\subsection{January 17\textsuperscript{th} \\ Creating R packages and Fitting Statistical Models}
  
This session focuses on the creation of R packages, a brief overview of
statistical simulation (as in the necessary components) as they relate to
creating R packages, and the use of model-fitting functions in this role.
  
\begin{itemize}
  \item Basics of writing R packages
  \item Overview of programming statistical simulation
  \item Working with the {\tt zelig2} function
  \item Constructing a statistical package
  \item Interfacing Zelig with external statistical packages
  \item Citation
\end{itemize}


% January 18th
\subsection{
  January 18\textsuperscript{th} \\
  Working with Data and Counterfactuals
}

This section focuses on the role of data and parameter simulation in the overall
process of statistical simulation.

\begin{itemize}
  \item Working with the {\tt setx} function
  \item Specifying counterfactuals
  \item Working with the {\tt param} function
  \item Simulating parameters of an external model
  \item Programming statistical bootstrapping
\end{itemize}


% January 19th
\subsection{January 19\textsuperscript{th} \\ Simulating Quantities of Interest}

This session focuses on the role of the actual simulation of \emph{quantities of interest}.
That is, through the combined work of {\bf fitting statistical models} and {\bf parameter
simulation}, we are able to produce values with significant meaning. Emphasis will be placed
on developing both the qualitative and quantitative aspects of the process as well as the 
general applicability of the procedure.

\begin{itemize}
	\item Working with the {\tt sim} function
  \item Understanding statistical simulation
  \item Simulating quantities of interest
  \item Making predictions
\end{itemize}


% January 20th
\subsection{
  January 20\textsuperscript{th} \\
  Overview and Problem-Solving Workshop
}

This session is devoted to the final aspects of R package creation: solving problems, 
beautifying-output and submitting packages to CRAN. As a result, students will spend most of this
day completing their statistical package and working out solutions to the inevitable pitfalls
that accompany developing statistical packages.

\begin{itemize}
	\item Review of the complete process of developing statistical packages
  \item Putting it all together
  \item Problem-solving workshop
  \item Submitting packages to CRAN
  \item What's next?
\end{itemize}

\end{document}
