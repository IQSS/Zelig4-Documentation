\documentclass{article}


\title{Programming Statistical Models with Zelig 4}
\author{Matt Owen}


\begin{document}


\maketitle


% Introduction
%
\section{Introduction}
\label{intro}

In this course, students will develop statistical models in the R programming 
language, using the Zelig 4 software package. Emphasis is placed on using 
generalized modeling techniques - linear regressions, simulations - to create 
predictive models that work with a large array of data-sets. Students will also 
learn the basic of developing R packages for distribution via CRAN 
(The Comprehensive R Network), a platform for sharing open-source statistical 
software on the internet.


% Prequisites
%
\section{Prerequisites}
\label{prereq}

\begin{enumerate}
  \item Basic Programming: conditional statements, loops and writing functions
  \item Basic R Programming: Sampling distributions and Linear Regressions
  \item Statistics: Descriptive statistics, linear regressions and bootstrapping
  \item Math: Matrix algebra
\end{enumerate}


% Schedule
%
\section{Schedule}
\label{schedule}

\begin{tabular}{ | p{.3\textwidth} | p{.7\textwidth} |}

  \hline

  % Row 1
  January 17\textsuperscript{th} & 
  \begin{itemize}
    \item Basics of writing R packages
    \item Overview of programming statistical simulation
    \item Constructing a statistical package
    \item Interfacing Zelig with external statistical packages
    \item Citation
  \end{itemize}
  \\ \hline

  % Row 2
  January 18\textsuperscript{th} & 
  \begin{itemize}
    \item Specifying counterfactuals 
    \item Working with the {\tt setx} object
    \item Simulating parameters of an external model
    \item Programming statistical bootstrapping
  \end{itemize}
  \\ \hline

  % Row 3
  January 19\textsuperscript{th} & 
  \begin{itemize}
    \item Understanding statistical simulation
    \item Simulating quantities of interest
    \item Making predictions
    \item Generating counterfactuals
  \end{itemize}
  \\ \hline

  % Row 4
  January 20\textsuperscript{th} & 
  \begin{itemize}
    \item Putting it all together
    \item Problem-solving workshop
    \item Submitting packages to CRAN!
  \end{itemize}
  \\ \hline

\end{tabular}


\end{document}
