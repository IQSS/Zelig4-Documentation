\documentclass{article}

\title{Zelig Formula Specification}
\author{Matt Owen}

\begin{document}

\maketitle

\section{Introduction}

The range of allowable formula in Zelig is often over-articulate. This allows
end-users and developers a wide-range of syntactically legal formula options.

The following document defines the specification for models to be submitted to
Zelig.


\section{Basic Formula}

The basic structure of a Zelig formula conforms to the following specifications:

\begin{enumerate}
  \item The left-hand side must be written in one of two formats:
    \begin{itemize}
      \item It must evaluate to a singular vector, or
      \item It must evaluate to a matrix with one column per outcome variable.
    \end{itemize}
  \item The left-hand side of the formula specifies the outcome (dependent) variables.
  \item The formula must be flat. That is, nested formula are considered
    illegal. For example, the formula {\tt y ~ a ~ b} is considered illegal.
  \item The left-hand side only supports "+", "*". ":", and "cbind",
  \item "cbind" can only be used as the first operation on the left-hand side of an equation.
  \item The right-hand side of the formula specifies the explanatory (independent) variables.
  \item The formula may be a list if it is describing multiple outcome variables.
\end{enumerate}





\end{document}
