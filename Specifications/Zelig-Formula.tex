\documentclass{article}

\usepackage{hyperref}

\title{Zelig Formula Specification}
\author{Matt Owen}

\newcommand{\tweedly}[0]{$\sim${ }}

\begin{document}

\maketitle

\section{Introduction}

The following is a technical specification for allowable types of model formula
objects that are, should be, and could be supported by Zelig Core. In
particular, ...



%
%
%
\section{Requirements}
\label{sec:req}

Zelig formulae need to be written generally enough to handle several types of
models:

\begin{itemize}

  \item A single response term specified within a single model equation

  \item A single response term specified across multiple model equations

  \item Multiple response terms specified within a single model equation

  \item Multiple response terms specified within multiple model equations
\end{itemize}


% \footnote{\tt y \TILDE a + b + c}
% \footnote{\tt cbind(x,y) \TILDE a + b + c}
% \footnote{\tt list(x \TILDE a + b + c, y \TILDE a + b + c}


%
%
%
\section{Existing Conventions in Zelig}
\label{sec:existing-zelig}

The above requirements specified in section \ref{sec:req}, have implementations
in Zelig (versions 3.5 and 4.0).


\subsection{Table}
% Table
{\noindent}\begin{tabular}{|l|l|l|}

  % Initial border
  \hline

  % Column Names (Row #1)
  & Single Response Term & Multiple Response Terms \\ \hline

  % Row #2
  Single Equation &
  {\tt y \tweedly a + b + c} & {\tt cbind(x, y) \tweedly a + b + c}
  \\ \hline

  % Row #3
  Multiple Equations & 
  {\tt list(x \tweedly a, y \tweedly b)} &
  {\tt list(x \tweedly a, y \tweedly b)} \\ \hline


\end{tabular}

\subsection{Technical Implementation}

The four types of formulae are craeted using three basic ingredients:

\begin{itemize}

  \item {\bf ``formula'' objects}, which allows R programs to evaluate the
    same formula in multiple contexts.

  \item {\bf the ``cbind'' function}, which allows users to tersely specify
    multiple response terms without.

  \item {\bf ``list'' objects}, which allow users to specify multiple
    model equations in an intuitive fashion.

\end{itemize}

Additionally, there is support for several tags within an



%
%
%
\section{Existing Conventions Outside of Zelig}
\label{sec:existing-elsewhere}

There are several existing conventions that extend the functionality of standard
formula objects:

\begin{itemize}

  \item Basic model formula designated using the \verb+~+ operator specifies a single
    model equation with a single response variable. Example: 
    {\tt y \tweedly a + b}

  \item Model formula designated using the \verb+cbind+ function in conjunction
    with the \verb+~+ operator specify multiple response terms for a single
    equation or set of predictor terms. Example: 
    {\tt cbind(x, y) \tweedly a + b + c}

  \item The ``Formula'', object found at
    \url{http://cran.r-project.org/web/packages/Formula/}, encapsulates the
    features of a standard ``formula'' object while extending its functionality.
    In particular, ``Formula'' objects allow for the specification of multiple
    response terms and equations without using a ``list'' object or a call to
    ``cbind''. Examples:
    \begin{itemize}
      \item {\tt Formula(y \tweedly a + b)}
      \item {\tt Formula(x | y \tweedly a + b)}
      \item {\tt Formula(y \tweedly a | b)}
      \item {\tt Formula(x | y \tweedly a | b)}
   \end{itemize}
   More on the ``Formula'' object specification can be found here:\\
   \url{http://cran.r-project.org/web/packages/Formula/index.html}

\end{itemize}





\section{Technical Specification for Basic Formula}

The basic structure of a Zelig formula conforms to the following specifications:

\begin{enumerate}

  \item The left-hand side must be written in one of two formats:

    \begin{itemize}

      \item It must evaluate to a singular vector \emph{or}

      \item It must evaluate to a matrix with one column per outcome variable.
        \footnote{
          This allows strict support for the {\tt cbind} function within the
          left-hand-side of a ``formula'' object
        }

    \end{itemize}

  \item The left-hand side of the formula specifies the outcome (dependent) variables.

  \item The formula must be flat. That is, nested formula are considered
    illegal. For example, the formula \begin{verbatim}y ~ a ~ b\end{verbatim} is considered illegal.

  \item The left-hand side only supports the following operators: +, *, :, | and cbind

  \item ``cbind'' can only be used as the first operation on the left-hand side of an equation.

  \item The right-hand side of the formula specifies the explanatory (independent) variables.

  \item The formula may be a list if it is describing multiple outcome variables.

\end{enumerate}



\section{Technical Specification for Basic Formula}



\end{document}
