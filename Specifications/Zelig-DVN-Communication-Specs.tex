\documentclass{article}

\title{Zelig-DVN Connectivity Specification}
\author{Matt Owen}

\begin{document}

\maketitle

\section{Introduction}
\label{intro}

The following is a proposed revision to the Zelig-DVN (Zelig-Dataverse) 
Communication specifications. 

\section{Necessary Components}
\label{Necessary-Components}

\begin{itemize}
  \item End-user
  \item DVN Website
  \item DVN
  \item Zelig Server
  \item Mechanism for spawning R processes
\end{itemize}

\section{Basic Structure}
\label{Basic Structure}

Typical scenario:

\begin{enumerate}
  \item User requests the analysis of a data-set, specifying:
    \begin{itemize}
      \item outcome variables
      \item explanatory variables
      \item statistical model
      \item dataset
      \item other parameters
    \end{itemize}

  \item Dataverse forwards the user-request to a Zelig server daemon. This can
    be essentially identical to the currently existing mechanism.

  \item The Zelig server hears the request, and:
    \begin{enumerate}
      \item Requests the data-set from the DVN ({\tt Zelig})
      \item Reads the columns, and classifies the type of data submitted.
        \footnote{That is, whether each column contains numeric, integer, or
          factor data. If integer or factor data is submitted, the number of
          unique results is recorded, to determine whether the data requires a
          multinomial model.
        } ({\tt ZeligDVN})
      \item The Zelig server checks whether the user request can be fulfilled.
        This involves:
        \begin{itemize}
          \item Determining whether the formula submitted by the user matches 
            the expected input format of the statistical model ({\tt ZeligDVN})
          \item Determining whether the data-set's columns contain values that
            matches the expected data-types of the statistical model
        \end{itemize}
      \item Returns responds to the DVN's request, specifying whether the model
        with the given parameters can be computed.
    \end{enumerate}
    
  \item The DVN requests the analysis to be performed

  \item The Zelig server hears the request, and:
    \begin{enumerate}
      \item Loads the necessary Zelig package
      \item Constructs the appropriate formula
      \item Loads the appropriate data from DVN
      \item Performs the analysis (using Zelig, setx and sim)
      \item Returns the analysis as a PNG file. This should include:
        \begin{itemize}
          \item The plot of the fitted model
          \item The plot of the simulated results
        \end{itemize}
      \item Additionally, plain-text results can be submitted for use with
        AJAX, if this is desired.
    \end{enumerate}
\end{enumerate}







\end{document}
